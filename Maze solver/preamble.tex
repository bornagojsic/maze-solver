%------------------------------------------------------
%
%   Mijenjati po potrebi, i to u vlastitom predavanju!
%
%   Za promjene u samom predlošku,
%   konzultirajte osobu zaduženu za održavanje.
%
%------------------------------------------------------

\documentclass[crop=false, class = scrartcl]{standalone}

\usepackage[utf8]{inputenc}
\usepackage[T1]{fontenc}
\usepackage[croatian]{babel}

\usepackage{mlmodern} % za fontove
 
\usepackage{csquotes} % za prave navodnike

\usepackage{subfiles} % da mogu različiti fileovi funkcionirat
\usepackage{amsmath, amssymb, amsthm} % hrpa matematičkih formula/znakova
\usepackage{enumitem} % formatiranje numeriranja

\usepackage{graphicx} % za slike
\graphicspath{ {./images/} }

\usepackage[dvipsnames]{xcolor} % za bojanje
\usepackage{tikz} % za najosnovnije slike

\usepackage{multirow} %
\usepackage{multicol} % za formatiranje naslova

\usepackage{fontawesome} % za ove unicode ikone

\usepackage[top=2cm,bottom=3cm,left=2cm,right=2cm]{geometry} % margine stranica

%\usepackage{biblatex}
%\addbibresource{literatura.bib}

\setlength{\parindent}{0cm} % da se odlomci ne uvlače
%\pagenumbering{gobble} % za nenumeriranje
\frenchspacing

\definecolor{orange_mnm}{HTML}{F37021}
\definecolor{grey_mnm}{HTML}{666666}

%Formatiranje poglavlja i potpoglavlja
\renewcommand*{\sectionformat}{\LARGE{\color{orange_mnm!80!}\thesection.}\enskip}
\renewcommand*{\subsectionformat}{\large{\color{orange_mnm!80!}\thesubsection.}\enskip}
\renewcommand*{\subsubsectionformat}{\large{\color{orange_mnm!80!}\thesubsubsection.}\enskip}

%Bitno za formatiranje naslova!
\usepackage{array}
\newcolumntype{L}[1]{>{\raggedright\let\newline\\\arraybackslash\hspace{0pt}}m{#1}}
\newcolumntype{C}[1]{>{\centering\let\newline\\\arraybackslash\hspace{0pt}}m{#1}}
\newcolumntype{R}[1]{>{\raggedleft\let\newline\\\arraybackslash\hspace{0pt}}m{#1}}

% TEOREMSKA OKRUZENJA:
\usepackage[framemethod = tikz]{mdframed} % za namještanje theorem tip environmenata
\usetikzlibrary{shadows}
\usepackage{thmtools} % za lakše editanje theorem tip 

\makeatletter
\define@key{thmdef}{mdthm}[{}]{%
\thmt@trytwice{\def\thmt@theoremdefiner{\mdtheorem[#1]}}{}}
\makeatother

\mdfdefinestyle{thmstyle}{
linewidth = 1.5pt,
linecolor = orange_mnm!90!,
hidealllines = true,
leftline = true,
innerleftmargin = 3pt,
innerrightmargin = 3pt,
frametitleaboveskip = 5pt,
frametitlebelowskip = 5pt,
frametitlerule = false,
frametitlebackgroundcolor = orange!5,
frametitlefont = \sffamily\bfseries,
shadow = true,
shadowsize = 4pt,
shadowcolor = black!5!
}

\declaretheoremstyle[
  headfont=\normalfont\itshape,
  spacebelow = 2pt,
  spaceabove = 12pt
]{italic}
		
\mdfdefinestyle{defstyle}{
linewidth = 1.5pt,
linecolor = RoyalBlue!90!green!,
hidealllines = true,
leftline = true,
innerleftmargin = 3pt,
innerrightmargin = 3pt,
frametitleaboveskip = 5pt,
frametitlebelowskip = 5pt,
frametitlerule = false,
frametitlebackgroundcolor = RoyalBlue!5,
frametitlefont = \sffamily\bfseries,
shadow = true,
shadowsize = 4pt,
shadowcolor = black!5!
}

\mdfdefinestyle{lemstyle}{
linewidth = 1.5pt,
linecolor = ForestGreen!90!black!,
hidealllines = true,
leftline = true,
innerleftmargin = 3pt,
innerrightmargin = 3pt,
frametitleaboveskip = 5pt,
frametitlebelowskip = 5pt,
frametitlerule = false,
frametitlebackgroundcolor = ForestGreen!5,
frametitlefont = \sffamily\bfseries,
shadow = true,
shadowsize = 4pt,
shadowcolor = black!5!
}

\mdfdefinestyle{oprstyle}{
linewidth = 1.5pt,
linecolor = WildStrawberry!90!black!,
hidealllines = true,
leftline = true,
innerleftmargin = 3pt,
innerrightmargin = 3pt,
frametitleaboveskip = 5pt,
frametitlebelowskip = 5pt,
frametitlerule = false,
frametitlebackgroundcolor = Salmon!10,
frametitlefont = \sffamily\bfseries,
shadow = true,
shadowsize = 4pt,
shadowcolor = black!5!
}

\declaretheoremstyle[
  headfont=\color{ForestGreen!80!black}\sffamily\bfseries,
  spacebelow = 2pt,
  spaceabove = 12pt
]{primjersty}

\declaretheoremstyle[
  headfont=\color{RawSienna!60!red!}\sffamily\bfseries,
  spacebelow = 6pt,
  spaceabove = 6pt
]{rjesenjesty}

%\declaretheorem[mdthm = {style = thmstyle}, name = {{\bfseries\sffamily\color{orange_mnm!30!orange!} \faInfoCircle} Teorem}, numberwithin = section]{teorem}
\declaretheorem[mdthm = {style = thmstyle}, name = {Teorem}, numberwithin = section]{teorem}
%\declaretheorem[mdthm = {style = thmstyle}, name = {{\bfseries\sffamily\color{orange_mnm!30!orange!} \faInfoCircle} Propozicija}, sibling = teorem]{propozicija}
\declaretheorem[mdthm = {style = thmstyle}, name = {Propozicija}, sibling = teorem]{propozicija}
%\declaretheorem[mdthm = {style = thmstyle}, name = {{\bfseries\sffamily\color{orange_mnm!30!orange!} \faInfoCircle} Korolar}, sibling = teorem]{korolar}
\declaretheorem[mdthm = {style = thmstyle}, name = {Korolar}, sibling = teorem]{korolar}
\declaretheorem[name = Dokaz, numbered = no, style = italic, qed = $\qedsymbol$]{dokaz}
%\declaretheorem[name = {\bfseries\sffamily {\color{RoyalBlue!30!blue!} \faInfoCircle} Definicija}, mdthm = {style = defstyle}, sibling = teorem]{definicija}
\declaretheorem[name = {Definicija}, mdthm = {style = defstyle}, sibling = teorem]{definicija}
%\declaretheorem[name = {{\bfseries\sffamily\color{ForestGreen!30!green!} \faInfoCircle} Lema}, mdthm = {style = lemstyle}, sibling = teorem]{lema}
\declaretheorem[name = {Lema}, mdthm = {style = lemstyle}, sibling = teorem]{lema}
\declaretheorem[name = {{\bfseries\sffamily\color{WildStrawberry!90!black!} \faExclamationTriangle} Oprez}, mdthm = {style = oprstyle}]{oprez}

\declaretheorem[name = Primjer, style = primjersty, parent = section]{primjer}
\declaretheorem[name = Zadatak, style = primjersty, sharenumber = primjer]{zadatak}
\declaretheorem[name = Rješenje, style = rjesenjesty, parent = section, qed = \qedsymbol]{rjesenje}

%Postavke za linkove:
\usepackage{hyperref}
\hypersetup{
    colorlinks=true,
    linkcolor=blue,
    filecolor=magenta,      
    urlcolor=red,
}

\urlstyle{same}